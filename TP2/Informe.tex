\documentclass[a4paper,12pt]{article}
\usepackage[spanish]{babel} % Soporte en español.
%\usepackage[latin1]{inputenc} % Caracteres con acentos.
\usepackage[utf8]{inputenc}
\usepackage{semantic}
\usepackage{listings}   %Importar codigo
%\usepackage{amsmath}
\usepackage{verbatim}
\usepackage{latexsym}
\usepackage{listingsutf8}
\usepackage{minted}

\lstset{ 
  inputencoding=utf8/latin1,
  basicstyle=\ttfamily \footnotesize,%\scriptsize, %\footnotesize,        % the size of the fonts that are used for the code
  breakatwhitespace=false,         % sets if automatic breaks should only happen at whitespace
  breaklines=false,                 % sets automatic line breaking
  captionpos=t,                    % sets the caption-position to bottom
  deletekeywords={...},            % if you want to delete keywords from the given language
  escapeinside={\%*}{*)},          % if you want to add LaTeX within your code
  extendedchars=true,              % lets you use non-ASCII characters; for 8-bits encodings only, does not work with UTF-8
  keepspaces=true,                 % keeps spaces in text, useful for keeping indentation of code (possibly needs columns=flexible)
  language=Haskell,                 % the language of the code
  %morekeywords={*,...},            % if you want to add more keywords to the set
  numbers=left,                    % where to put the line-numbers; possible values are (none, left, right)
  numbersep=10pt,                   % how far the line-numbers are from the code
  showspaces=false,                % show spaces everywhere adding particular underscores; it overrides 'showstringspaces'
  showstringspaces=false,          % underline spaces within strings only
  showtabs=false,                  % show tabs within strings adding particular underscores
  stepnumber=1,                    % the step between two line-numbers. If it's 1, each line will be numbered
  tabsize=1,                       % sets default tabsize to 2 spaces
  title=\lstname                   % show the filename of files included with \lstinputlisting; also try caption instead of title
}

\begin{document}

\title{Análisis de Lenguajes de Programación\\ 
        Trabajo Práctico 2}
\author{Lauro Figueroa}
\date{2 de Octubre de 2014}
\maketitle

\pagenumbering{gobble} 
\newpage
\pagenumbering{arabic}

\section*{Ejercicio 2}

\subsection*{Gramática extendida del $\lambda$-cálculo}

\iffalse
\begin{minted}{text}
Term ::= Var | '\' Var '.' Term | Term Term | '(' Term ')' 
Var ::= 'a' | ... | 'z'
\end{minted}


\begin{minted}{text}
Term ::= NApp | Term NApp
NApp ::= Var | '\' Var '.' Term | '(' Term ')'
Var ::= 'a' | ... | 'z'
\end{minted}
\fi

\begin{minted}{text}
Term ::= NApp | Term NApp
NAbs ::= Atom | NAbs NApp
NApp ::= Atom | Abs
Atom ::= Var  | '(' Term ')'
Abs  ::= '\ ' Vars '.' Term 
Vars ::= Var | Var Vars
Var  ::= 'a' | ... | 'z'
\end{minted}

La gramática presentada cumple con las convenciones que se usan para escribir
$\lambda$-términos pero  presenta el problema de recursión a izquierda.
Una solución es transformar la gramática a una equivalente donde se elimine la
recursión a izquierda.

\subsection*{Transformación de la gramática del $\lambda$-cálculo}

\begin{minted}{text}
Term ::= NApp | Term NApp
NAbs ::= Atom NAbs'
NAbs'::= e    | NApp NAbs'
NApp ::= Atom | Abs
Atom ::= Var  | '(' Term ')'
Abs  ::= '\' Vars '.' Term 
Vars ::= Var | Var Vars
Var  ::= 'a' | ... | 'z'
\end{minted}

Un nuevo problema surgió, al transformar la gramática a otra que no tenga
recursión a izquierda, la misma perdió la asociatividad a izquierda necesaria para 
respetar las convenciones, este problema lo resuelvo en la implementación 
donde busco los Atom y los asocio de manera correcta con una función auxiliar,
evitando así el problema.

\section*{Ejecicio 7}
La implementación ofrecida en $\lambda$-cálculo de la función raíz cuadrada es parcial.


\newpage
\section*{\centering Codigo fuente}
\subsection*{Untyped.hs}
\inputminted{haskell}{Untyped.hs}
\subsection*{Sqrt.lam}
\inputminted{haskell}{Sqrt.lam}
%\lstinputlisting{Untyped.hs}
%\lstinputlisting{Sqrt.lam}
\end{document}
