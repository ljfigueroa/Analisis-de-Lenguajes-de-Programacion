\documentclass[a4paper,12pt]{article}

\usepackage[left=1in,right=1in,top=1.2in,bottom=1.2in]{geometry}

\usepackage{url,graphicx,tabularx,array,geometry}
\usepackage{fontspec}
\usepackage{polyglossia}
\usepackage{stmaryrd}
\usepackage{amsmath}
\usepackage{amssymb}
\usepackage{amsthm}
\usepackage{amsfonts}
\usepackage{nccmath}
\usepackage{minted}


\setmainlanguage{spanish}

\setlength{\parskip}{1ex} %--skip lines between paragraphs
\setlength{\parindent}{0pt} %--don't indent paragraphs


\newtheorem{theorem}{Theorem}%[section]
\newtheorem{lemma}[theorem]{Lemma}
\newtheorem{conj}[theorem]{Conjecture}

\begin{document}

\title{Análisis de Lenguajes de Programación\\ 
        Trabajo Práctico 4}
\author{Lauro Figueroa}
\date{13 de Noviembre de 2014}
\maketitle

\pagenumbering{gobble} 
\newpage
\pagenumbering{arabic}

\iffalse
\begin{fleqn}
\begin{align*}
    & \textrm{return } x \gg=  f
\\ = \;    & \langle \text{definición de return} \rangle
\end{align*}
\end{fleqn}
\fi

\section*{Ejercicio 1. a)}
Para poder probar que State es una monada, debo probar que $\gg=$ y $return$ cumplen
las leyes de las mónadas.

% ---- monda.1--------
\begin{proof}
Ley monad.1
\begin{center}
 return x $\gg=$ f = f x
\end{center}
%\end{proof}

%\begin{proof}
\hfill
\begin{fleqn}
\begin{equation} 
\begin{split}
  & \text{return x $\gg=$ f }\\
  & =\langle \mbox{Definición de return} \rangle \\
  & State \ (\backslash s \rightarrow (x,s)) \gg= f \\
  & =\langle \mbox{Definición de}\gg= \rangle \\
  & State \ (\backslash s' \rightarrow \mbox{let} \; (y,s'' ) = \mbox{runState} \ (State \ (\backslash s \to (x,s))) \ s' \\
  & \hspace{70pt} \mbox{in  runState} \ (f \ y) \ s'')\\
  & =\langle \mbox{runState}.State \ = \ \mbox{Id}  \rangle \\
  & State \ (\backslash s' \rightarrow \mbox{let} \; (y,s'') = (\backslash s \to (x,s)) \ s' \\
  & \hspace{70pt} \mbox{in  runState} \ (f \ y) \ s'')\\
  & =\langle \beta \mbox{-contracción}  \rangle \\
  & State \ (\backslash s' \rightarrow \mbox{let} \; (y,s'' ) = (x,s') \\
  & \hspace{70pt} \mbox{in  runState} \ (f \ y) \ s'')\\
  & =\langle \mbox{Sustitución por let}  \rangle \\
  & State \ (\backslash s' \rightarrow \mbox{runState} \ (f \ x) \ s')\\
  & =\langle \mbox{Extensionalidad}  \rangle \\
  & State \ (\mbox{runState} \ (f \ x))\\
  & =\langle \mbox{Composición}  \rangle \\
  & State.\mbox{runState} \ (f \ x)\\
  & =\langle State.\mbox{runState} \ = \ \mbox{Id}  \rangle \\
  & f \ x\\
\end{split}
\end{equation}
\end{fleqn}
\end{proof}

% ---- monda.2--------
\newpage
\begin{proof}
Ley monad.2
\begin{center}
 t $\gg=$ return \; = \ t
\end{center}

La clase Monad establece que ($\gg=$) :: Monad m $\Rightarrow$ m a $\to$ (a $\to$ m b) $\to$ m b,
entonces t :: m a y  puedo escribir como t = $State$ h para algún h. Luego

\begin{fleqn}
\begin{equation} 
\begin{split}
  & State \ \mbox{h} \gg= \text{return}\\
  & =\langle \mbox{Definición de}\gg= \rangle \\
  & State \ (\backslash s \to \mbox{let} \; (x,s') = \mbox{runState} \ (State \ \mbox{h}) \ s \\
  & \hspace{67pt} \mbox{in  runState} \ (\mbox{return} \ x) \ s')\\
  & =\langle \mbox{runState}.State \ = \ \mbox{Id}  \rangle \\
  & State \ (\backslash s \to \mbox{let} \; (x,s') = \mbox{h} \ s \\
  & \hspace{67pt} \mbox{in  runState} \ (\mbox{return} \ x) \ s')\\
  & =\langle \mbox{Definición return}  \rangle \\
  & State \ (\backslash s \to \mbox{let} \; (x,s') = \mbox{h} \ s \\
  & \hspace{67pt} \mbox{in  runState} \ (State \ (\backslash s''\to (x,s''))) \ s')\\
  & =\langle \mbox{runState}.State \ = \ \mbox{Id}  \rangle \\
  & State \ (\backslash s \to \mbox{let} \; (x,s') = \mbox{h} \ s\\
  & \hspace{67pt} \mbox{in} \ (\backslash s''\to (x,s'')) \ s')\\
  & =\langle \beta \mbox{-contracción}  \rangle \\
  & State \ (\backslash s \to \mbox{let} \; (x,s') = \mbox{h} \ s\\
  & \hspace{67pt} \mbox{in} \ (x,s'))\\
  & =\langle \mbox{Sustitución por let}  \rangle \\
  & State \ (\backslash s \to  \mbox{h} \ s)\\
  & =\langle \mbox{Extensionalidad}  \rangle \\
  & State \ \mbox{h}
\end{split}
\end{equation}
\end{fleqn}
\end{proof}

% ---- monda.3--------
\newpage
\begin{proof}
Ley monad.3
\begin{center}
 (t $\gg=$ f) $\gg=$ g \; = \; t $\gg=$ ($\backslash$x $\to$ f x  $\gg=$ g)
\end{center}

La clase Monad establece que ($\gg=$) :: Monad m $\Rightarrow$ m a $\to$ (a $\to$ m b) $\to$ m b,
entonces t :: m a y  puedo escribir como t = $State$ h para algún h. Luego

\begin{fleqn}
\begin{equation} 
\begin{split}
  & (State \ \mbox{h} \gg= f) \gg= g\\
  & =\langle \mbox{Definición de}\gg= \rangle \\
  & (State \ (\backslash s \to \mbox{let} \; (x,s') = \mbox{runState} \ (State \ \mbox{h}) \ s \\
  & \hspace{71pt} \mbox{in  runState} \ (f \ x) \ s')) \gg=  \ g \\
  & =\langle \mbox{runState}.State \ = \ \mbox{Id}  \rangle \\
  & (State \ (\backslash s \to \mbox{let} \; (x,s') =  \mbox{h} \ s \\
  & \hspace{71pt} \mbox{in  runState} \ (f \ x) \ s')) \gg=  \ g \\
  & =\langle \mbox{Definición} \gg=  \rangle \\
  & State \ (\backslash r \to \mbox{let} \; (y,r') = \mbox{runState} \ (State \ (\backslash s \to \mbox{let} \; (x,s') = \mbox{h} \ s \\
  & \hspace{248pt} \mbox{in  runState} \ (f \ x) \ s')) \ r \\
  & \hspace{68pt} \mbox{in  runState} \ (g \ y) \ r' )\\
  & =\langle \mbox{runState}.State \ = \ \mbox{Id}  \rangle \\
  & State \ (\backslash r \to \mbox{let} \; (y,r') = (\backslash s \to \mbox{let} \; (x,s') = \mbox{h} \ s \\
  & \hspace{163pt} \mbox{in  runState} \ (f \ x) \ s') \ r\\
  & \hspace{68pt} \mbox{in  runState} \ (g \ y) \ r' )\\
  & =\langle \beta \mbox{-contracción}  \rangle \\
  %& =\langle \mbox{Abstracción}  \rangle \\
  & State \ (\backslash r \to \mbox{let} \; (y,r') = \mbox{let} \; (x,s') = \mbox{h} \ r \\
  & \hspace{128pt} \mbox{in  runState} \ (f \ x) \ r\\
  & \hspace{65pt} \mbox{in  runState} \ (g \ y) \ r' )\\
  & =\langle \mbox{Sustitución por let}  \rangle \\
  & State \ (\backslash r \to \mbox{let} \; (x,s') = \mbox{h} \ r \\
  & \hspace{83pt} (y,r') = \mbox{runState} \ (f \ x) \ s'\\
  & \hspace{68pt} \mbox{in  runState} \ (g \ y) \ r' )\\
  & =\langle \mbox{Sustitución por let}  \rangle \\
  & State \ (\backslash r \to \mbox{let} \; (x,s') = \mbox{h} \ r \\
  & \hspace{68pt} \mbox{in let} \ (y,r') = \mbox{runState} \ (f \ x) \ s'\\
  & \hspace{82pt} \mbox{in  runState} \ (g \ y) \ r' )\\
\end{split}
\end{equation}
\end{fleqn}

\begin{fleqn}
\begin{equation*} 
\begin{split}
  & =\langle \mbox{Abstracción}  \rangle \\
  & State \ (\backslash r \to \mbox{let} \; (x,s') = \mbox{h} \ r \\
  & \hspace{68pt} \mbox{in} \ (\backslash s \to \mbox{let} \ (y,r') = \mbox{runState} \ (f \ x) \ s\\
  & \hspace{116pt} \mbox{in  runState} \ (g \ y) \ r' ) \ s')\\
  & =\langle \mbox{runState.} State \ = \ \mbox{Id}  \rangle \\
  & State \ (\backslash r \to \mbox{let} \; (x,s') = \mbox{h} \ r \\
  & \hspace{68pt} \mbox{in runState} \ (State \ (\backslash s \to \mbox{let} (y,r') = \mbox{runState} \ (f \ x) \ s\\
  & \hspace{202pt} \mbox{in  runState} \ (g \ y) \ r' ) \ s'))\\
  & =\langle \mbox{Definición de}\gg= \rangle \\
  & State \ (\backslash r \to \mbox{let} \; (x,s') = \mbox{h} \ r \\
  & \hspace{68pt} \mbox{in runState} \ (f \ x \gg= \ g) \ s')\\
  & =\langle \mbox{Abstracción}  \rangle \\
  & State (\backslash r \to \mbox{let} \; (x,s') = \mbox{h} \ r \\
  & \hspace{62pt} \mbox{in runState} \ ((\backslash x \to (f \ x \gg= \ g) \ x)\ s')\\
  & =\langle \mbox{Definición de}\gg= \rangle \\
  & State \ \mbox{h} \gg= (\backslash x \to (f \ x \gg= \ g))\\
\end{split}
\end{equation*}
\end{fleqn}
\end{proof}


Por lo tanto como el constructor $State$ cumple con las tres ecuaciones de mónadas, es una mónada.


\newpage
\section*{Codigo fuente}

\subsection*{Eval1.hs}
\inputminted[fontsize=\footnotesize]{haskell}{Eval1.hs}
\subsection*{Eval2.hs}
\inputminted[fontsize=\footnotesize]{haskell}{Eval2.hs}
\subsection*{Eval3.hs}
\inputminted[fontsize=\footnotesize]{haskell}{Eval3.hs}

\end{document}
