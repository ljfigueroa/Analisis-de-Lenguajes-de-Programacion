% pdflatex -shell-escape Informe.tex

\documentclass[a4paper,12pt]{article}
\usepackage[spanish]{babel} % Soporte en español.
%\usepackage[latin1]{inputenc} % Caracteres con acentos.
\usepackage[utf8]{inputenc}
\usepackage{semantic}
\usepackage{listings}   %Importar codigo
%\usepackage{amsmath}
\usepackage{verbatim}
\usepackage{latexsym}
\usepackage{proof}
\lstset{ %
  basicstyle=\ttfamily \footnotesize%\scriptsize, %\footnotesize,        % the size of the fonts that are used for the code
  breakatwhitespace=false,         % sets if automatic breaks should only happen at whitespace
  breaklines=true,                 % sets automatic line breaking
  captionpos=t,                    % sets the caption-position to bottom
  deletekeywords={...},            % if you want to delete keywords from the given language
  escapeinside={\%*}{*)},          % if you want to add LaTeX within your code
  extendedchars=true,              % lets you use non-ASCII characters; for 8-bits encodings only, does not work with UTF-8
  keepspaces=true,                 % keeps spaces in text, useful for keeping indentation of code (possibly needs columns=flexible)
  language=Haskell,                 % the language of the code
  morekeywords={*,...},            % if you want to add more keywords to the set
  numbers=left,                    % where to put the line-numbers; possible values are (none, left, right)
  numbersep=10pt,                   % how far the line-numbers are from the code
  showspaces=false,                % show spaces everywhere adding particular underscores; it overrides 'showstringspaces'
  showstringspaces=false,          % underline spaces within strings only
  showtabs=false,                  % show tabs within strings adding particular underscores
  stepnumber=1,                    % the step between two line-numbers. If it's 1, each line will be numbered
  tabsize=1,                       % sets default tabsize to 2 spaces
  title=\lstname                   % show the filename of files included with \lstinputlisting; also try caption instead of title
}

\begin{document}

\title{Análisis de Lenguajes de Programación\\ 
        Trabajo Práctico 1}
\author{Lauro Figueroa}
\date{4 de Septiembre de 2014}
\maketitle
\newpage

%\newpage

\section{Ejercicio 2.2.1}
Para poder incluir a la operación $"?:"$ agrego la siguiente linea a la sintaxis abstracta de LIS:\\

% \langle y \rangle
%\begin{verbatim}
\textless $intexp$ \textgreater $::=$ \quad $|$ \textless $boolexp$ \textgreater\ $?$ \textless $intexp$ \textgreater\ $:$ \textless $intexp$ \textgreater \\
%\end{verbatim}

Además tengo que añadir la siguiente linea a la sintaxis concreta:\\

%\begin{verbatim}
\textless $intexp$ \textgreater $::=$ \quad $|$ \textless $boolexp$ \textgreater\ '?' \textless\ $intexp$ \textgreater\ ':' \textless $intexp$ \textgreater
%\end{verbatim}


\section{Ejercicio 2.4.1}

Agrego la siguiente linea a la semántica denotacional de expreciones enteras para poder incluir el
operador ternario:

\[
|[{p\; ?\ e_1 : e_2}|]_{intexp^{\sigma }} = 
\left\{
  \begin{array}{l l}
    |[e_1|]_{intexp^{\sigma }} & \quad cuando \ |[p|]_{boolexp^{\sigma }} = true\\
    |[e_2|]_{intexp^{\sigma }} & \quad cuando \ |[p|]_{boolexp^{\sigma }} = false\\
  \end{array} 
\right.
\]

\section{Ejercicio 2.5.1}

Para probar que que la relación de evaluación $\leadsto$ en un paso es determinista, debo 
probar que si $ \gamma \leadsto \gamma'$ y $\gamma \leadsto \gamma''$ entonces $\gamma' = \gamma''$.
Donde cada $\gamma$ es un estado (una configuración terminal) o un comando junto con un conjunto de 
estados (una configuración no terminal).\\

Dem/ Por inducción en la derivación de $ \gamma \leadsto \gamma'$. En cada paso de la derivación
asumo que la regla vale para el resultado de derivaciones mas chicas y procedo por el análisis
de la ultima regla de evaluación usada en la derivación.\\

%%%%%%%%
Si la ultima regla usada en la derivación de $\gamma \leadsto \gamma'$ es ASS entonces sabemos que 
cada configuración es de la forma:

\begin{itemize}
  \item $\gamma$ = \textless v ::= e, $\sigma$\textgreater
  \item $\gamma'$ =  $[ \sigma | v: |[e|]_{intexp^{\sigma}}]$ 
\end{itemize}

Luego vemos que la ultima regla aplicada en la derivación de $\gamma \leadsto \gamma''$ 
no puede haber sido otra diferente de ASS por la forma de $\gamma$ que solo admite 
aplicar esa única regla de inferencia. Entonces claramente $\gamma' = \gamma''$.\\


%%%%%%%%
Si la ultima regla usada en la derivación de $\gamma \leadsto \gamma'$ es SKIP entonces sabemos que 
cada configuración es de la forma:

\begin{itemize}
  \item $\gamma$ = \textless skip, $\sigma$\textgreater
  \item $\gamma'$ =  $\sigma$ 
\end{itemize}

Luego vemos que la ultima regla aplicada en la derivación de $\gamma \leadsto \gamma''$ 
no puede haber sido otra diferente de SKIP por la forma de $\gamma$ que solo admite 
aplicar esa única regla de inferencia. Entonces claramente $\gamma' = \gamma''$.\\

%%%%%%%%
Si la ultima regla usada en la derivación de $\gamma \leadsto \gamma'$ es $SEQ_{1}$ entonces sabemos que 
cada configuración es de la forma:

\begin{itemize}
  \item $\gamma$ = \textless $c_{0};c_{1}$, $\sigma$\textgreater
  \item $\gamma'$ =  \textless $c_{1}$, $\sigma'$\textgreater
  \item \textless $c_{0}$, $\sigma$\textgreater \ $\leadsto$\ $\sigma'$
\end{itemize}

Luego vemos que la ultima regla aplicada en la derivación de $\gamma \leadsto \gamma''$ 
puede haber sido $SEQ_{2}$ o $SEQ_{1}$ por la forma de $\gamma$ que solo admite 
aplicar una de estas reglas de inferencia. Supongo que en la derivación de $\gamma''$
se obtuvo a partir de $SEQ_{2}$, luego 

\begin{itemize}
  \item $\gamma$ = \textless $c_{0};c_{1}$, $\sigma$\textgreater (ya lo sabíamos)
  \item $\gamma''$ =  \textless ${c^{'}_{0}};c_{1}$, $\sigma'$\textgreater
  \item \textless $c_{0}$, $\sigma$\textgreater\ $\leadsto$ \textless ${c_{0}^{'}}$, $\sigma'$\textgreater
\end{itemize}

Llegamos a una contradicción porque por HI sabemos que para toda expresión mas pequeña vale la 
propiedad, pero \  \textless $c_{0}$, $\sigma$\textgreater\ $\leadsto$ \textless ${c_{0}^{'}}$, $\sigma'$\textgreater \  \ y \\\textless $c_{0}$, $\sigma$\textgreater\ $\leadsto$  $\sigma'$ y 
  siendo
$\sigma'$ $\neq$ \textless ${c_{0}^{'}}$, $\sigma'$\textgreater . Esta contradicción provino de suponer
 que la ultima regla aplicada fue $SEQ_{2}$. Entonces se obtuvo $\gamma''$ de la regla de 
inferencia $SEQ_{1}$. Por lo tanto $\gamma' = \gamma''$.\\

%%%%%%%%
Si la ultima regla usada en la derivación de $\gamma \leadsto \gamma'$ es $SEQ_{2}$, la prueba es 
análoga a la  de $SEQ_{1}$.\\

\newpage
%%%%%%%%
Si la ultima regla usada en la derivación de $\gamma \leadsto \gamma'$ es $IF_{1}$ entonces sabemos que 
cada configuración es de la forma:

\begin{itemize}
  \item $\gamma$ = \textless if $b$ then $c_{0}$ else $c_{1}$, $\sigma$\textgreater
  \item $\gamma'$ =  \textless $c_{0}$, $\sigma$\textgreater
  \item $|[b|]_{boolexp^{\sigma}}$ = true 
\end{itemize}

Luego vemos que la ultima regla aplicada en la derivación de $\gamma \leadsto \gamma''$ 
puede haber sido $IF_{2}$ o $IF_{1}$ por la forma de $\gamma$ que solo admite 
aplicar una de estas reglas de inferencia, pero como $|[b|]_{boolexp^{\sigma}}$ = true, solo 
se puede haber sido $IF_{1}$. Por lo tanto $\gamma' = \gamma''$.\\

%%%%%%%%
Si la ultima regla usada en la derivación de $\gamma \leadsto \gamma'$ es $IF_{2}$, la prueba es 
análoga a la  de $IF_{1}$.\\


%%%%%%%%
Si la ultima regla usada en la derivación de $\gamma \leadsto \gamma'$ es $WHILE_{1}$ entonces 
sabemos que cada configuración es de la forma:

\begin{itemize}
  \item $\gamma$ = \textless while $b$ do $c$, $\sigma$\textgreater
  \item $\gamma'$ = \textless $c$; while $b$ do $c$, $\sigma$\textgreater 
  \item $|[b|]_{boolexp^{\sigma}}$ = true 
\end{itemize}

Luego vemos que la ultima regla aplicada en la derivación de $\gamma \leadsto \gamma''$ 
puede haber sido $WHILE_{2}$ o $WHILE_{1}$ por la forma de $\gamma$ que solo admite 
aplicar una de estas reglas de inferencia, pero como $|[b|]_{boolexp^{\sigma}}$ = true, solo 
se puede haber sido $WHILE_{1}$. Por lo tanto $\gamma' = \gamma''$.\\

%%%%%%%%
Si la ultima regla usada en la derivación de $\gamma \leadsto \gamma'$ es $WHILE_{2}$, la prueba es 
análoga a la  de $WHILE_{1}$.\\


\newpage
\section{Ejercicio 2.5.2}

Para construir el árbol de prueba, le asigne a cada configuración una letra para aliviar la lectura.\\

\noindent $O =$ \textless $x ::= x+1$; if $x > 0$ then $skip$ else $x ::= x-1$, $[\sigma|x:0]$\textgreater\\
$F =$ $[\sigma|x:1]$\\
$A =$  \textless $x::= x+1$, $[\sigma|x:0]$\textgreater\ $\leadsto$ $[\sigma|x:1]$\\
$I =$ \textless if $x > 0$ then $skip$ else $x ::= x-1$, $[\sigma|x:1]$\textgreater\\
$B =$ $|[x > 0|]_{boolexp^{[\sigma|x:1]}}$ = true\\
$S =$ \textless skip,$[\sigma|x:1]$\textgreater\\
%    \longrightarrow

$$
\vcenter{
\infer[TR_{2}]{O \leadsto^{*} F}{
    \infer[TR_{1}]{O\leadsto^{*}I}{
        \infer[SEQ_{1}]{O \leadsto I}{ 
            \infer[ASS]{A}{}
        }
    }
    &
    \infer[TR_{2}]{I \leadsto^{*} F}{
        \infer[TR_{1}]{I \leadsto^{*} S}{
            \infer[IF_{1}]{I \leadsto S}{B}
        }
        & 
        \infer[TR_{1}]{S \leadsto^{*} F}{
            \infer[SKIP]{S \leadsto F} {}
        }
    }
}
}
$$

\section{Ejercicio 2.5.6}

Reglas de la semántica operacional para until: \\

$$
\vcenter{
\infer{(repeat \ c \ until \ b,\sigma) \leadsto (c,\sigma)}{|[b|]_{boolexp^{\sigma}} = true}
}{}
\qquad
\vcenter{
\infer{(repeat\ c\ unti \ b,\sigma) \leadsto (c;repeat\ c\ until\ b,\sigma)}{|[b|]_{boolexp^{\sigma}} = false}{}
}
$$
\\
Para extender la gramática abstracta de LIS agrego la siguiente regla a la sintaxis abstracta:\\

\textless $comm$\textgreater $::=$ \quad $|$ repeat \textless $intexp$ \textgreater\ until \textless $boolexp$ \textgreater \\

\newpage
\lstinputlisting{AST.hs}
\lstinputlisting{Eval1.hs}
\lstinputlisting{Eval2.hs}
\lstinputlisting{Eval3.hs}
\end{document}
