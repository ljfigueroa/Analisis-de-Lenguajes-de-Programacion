% pdflatex -shell-escape Informe.tex
\documentclass[a4paper,12pt]{article}
\usepackage[spanish]{babel} % Soporte en español.
%\usepackage[latin1]{inputenc} % Caracteres con acenshortrightarrows.
\usepackage[utf8]{inputenc}
\usepackage{semantic}
\usepackage{listings}   %Importar codigo
%\usepackage{amsmath}
\usepackage{verbatim}
\usepackage{latexsym}
\usepackage{listingsutf8}
\usepackage{minted}
\usepackage{proof}
\usepackage{ stmaryrd }
\usepackage{ tipa }

\lstset{ 
  inputencoding=utf8/latin1,
  basicstyle=\ttfamily \footnotesize,%\scriptsize, %\footnotesize,        % the size of the fonts that are used for the code
  breakatwhitespace=false,         % sets if aushortrightarrowmatic breaks should only happen at whitespace
  breaklines=false,                 % sets aushortrightarrowmatic line breaking
  captionpos=t,                    % sets the caption-position shortrightarrow botshortrightarrowm
  deletekeywords={...},            % if you want shortrightarrow delete keywords from the given language
  escapeinside={\%*}{*)},          % if you want shortrightarrow add LaTeX within your code
  extendedchars=true,              % lets you use non-ASCII characters; for 8-bits encodings only, does not work with UTF-8
  keepspaces=true,                 % keeps spaces in text, useful for keeping indentation of code (possibly needs columns=flexible)
  language=Haskell,                 % the language of the code
  %morekeywords={*,...},            % if you want shortrightarrow add more keywords shortrightarrow the set
  numbers=left,                    % where shortrightarrow put the line-numbers; possible values are (none, left, right)
  numbersep=10pt,                   % how far the line-numbers are from the code
  showspaces=false,                % show spaces everywhere adding particular underscores; it overrides 'showstringspaces'
  showstringspaces=false,          % underline spaces within strings only
  showtabs=false,                  % show tabs within strings adding particular underscores
  stepnumber=1,                    % the step between two line-numbers. If it's 1, each line will be numbered
  tabsize=1,                       % sets default tabsize shortrightarrow 2 spaces
  title=\lstname                   % show the filename of files included with \lstinputlisting; also try caption instead of title
}

\begin{document}

\title{Análisis de Lenguajes de Programación\\ 
        Trabajo Práctico 3}
\author{Lauro Figueroa}
\date{9 de Octubre de 2014}
\maketitle

\pagenumbering{gobble} 
\newpage
\pagenumbering{arabic}

\section*{Ejercicio 1}

El término S esta definido de la siguiente forma: 
\begin{minted}{text}
   def S = \x:B->B->B.\y:B->B.\z:B.(x z) (y z)
\end{minted}

Dado S, le corresponde el siguiente árbol de derivación, donde defino $\Gamma$ como 
$x: B \shortrightarrow B\shortrightarrow B, y: B\shortrightarrow B, z: B $
para alivianar la notación del mismo.\\

$$
\vcenter{
\infer[T-Abs]{\vdash \lambda x\colon B \shortrightarrow B\shortrightarrow B.  \lambda y\colon B\shortrightarrow B.\lambda z:B.(x z) (y z) \colon (B \shortrightarrow B \shortrightarrow B) \shortrightarrow (B \shortrightarrow B) \shortrightarrow B \shortrightarrow B}{
  \infer[T-Abs]{x\colon B \shortrightarrow B\shortrightarrow B \vdash \lambda y\colon B\shortrightarrow B.\lambda z:B.(x z) (y z) \colon (B \shortrightarrow B) \shortrightarrow B \shortrightarrow B}{
    \infer[T-Abs]{x\colon B \shortrightarrow B\shortrightarrow B,y\colon B\shortrightarrow B \vdash \lambda z:B.(x z) (y z) \colon B \shortrightarrow B }{
        \infer[T-App]{\Gamma \vdash  (x z) (y z)\colon B}{ 
              \infer[T-App]{ \Gamma \vdash (x z)\colon B \shortrightarrow B} {
                \infer[T-Var]{\Gamma \vdash x\colon B \shortrightarrow B\shortrightarrow B} {x\colon B \shortrightarrow B\shortrightarrow B \in \Gamma}
                &
                \infer[T-Var]{\Gamma \vdash z\colon B} {x\colon B \in \Gamma}
              }
              &
              \infer[T-App]{\Gamma \vdash (y z)\colon B} {
                \infer[T-Var]{\Gamma \vdash y\colon B\shortrightarrow B} {y\colon B\shortrightarrow B \in \Gamma} {}
                &
                \infer[T-Var]{\Gamma \vdash z\colon B} {x\colon B \in \Gamma} {} 
              }
        }
    }
  }
  }
}
$$


\section*{Ejercicio 2}

La función infer retorna un valor de tipo $ Either \ String \ Type $ para que
el manejo de errores sea mas simple. Por ejemplo, en el caso que infiera un error de tipos devuelve 
 una cadena tipo $String$ pero en el caso opuesto, donde el tipo del termino es inferido de 
manera exitosa, retorna algo de tipo $Type$ y como ambos son distintos se debe usar un tipo de
datos que nos deje representar valores con dos posibilidades, por estas razones se usa 
$ Either \ String \ Type $ como tipo de retorno de infer. \\ \\
La función $\gg=$ se encarga de propagar los errores y de guardar el último tipo inferido. Su 
funcionamiento esta determinado por $either \ Left \ f \ v$ donde $v$ es un valor de tipo 
$ Either \ String \ Type $ y f una función de $Type\to Either\ String\ Type$.
Como $Left$ y $f$ son funciones que tienen el mismo tipo de
retorno es el mismo tipo, el valor retornado por $\gg=$ es la apicacion de $Left$ sobre la 
$string$ de $v$, que solo preserva su valor y  en cambio $f$ altera el valor de $type$. 
Resultando en la devolución de un valor donde, si había ocurrido un error se sigue manteniendo
la información y si se pudo inferir el tipo de manera correcta se actualiza el valor del tipo.



\section*{Ejercicio 5}

Dado el siguiente término  (let z = (($\lambda$x : $B$. x) as $B \to B$) in z) as $B \to B$ le
corresponde el siguiente árbol de derivación de tipos:

$$
\vcenter {
\infer[E-Ascribe] {\vdash (let\ z = ((\lambda x : B. x)\ as\ B \to B)\ in\ z)\ as\ B \to B : B\to B} {
  \infer[T-Let] {\vdash let\ z = ((\lambda x : B. x)\ as\ B \to B)\ in\ z: B \to B} {     
    \infer[E-Ascribe] {\vdash  ((\lambda x : B. x)\ as\ B \to B : B\to B} {
      \infer[T-Abs] {\vdash ((\lambda x : B. x)\ : \ B \to B} {
        \infer[T-Var] {x:B \vdash x:B} {x:B \in x:B}
        }
      }
    & 
    \infer[T-Var] {z:B\to B \vdash z:B\to B} {z:B\to B \in z:B\to B}
    }
  } 
}
$$

El mismo término evaluado en el intérprete, tipa correctamente como se puede ver a continuación:


\begin{minted}{text}
ST> :type (let z = ((\x : B. x) as B -> B) in z) as B -> B
B -> B
\end{minted} 


\section*{Ejercicio 7}

Sean las siguientes reglas de evaluación para los pares donde el criterio es reducir 
por completo $t_{1}$ para luego proceder a reducir  $t_{2}$:

$$
\vcenter {
\infer[E-Tup_{1}] {(t_{1}, t_{2}) \to (t_{1}', t_{2})} {t_{1} \to t_{1}'}}
\qquad 
\vcenter {
\infer[E-Tup_{2}] {(v, t_{2}) \to (v, t_{2}')} {t_{2} \to t_{2}'}}
$$
\\

\section*{Ejercicio 9}

Dado el siguiente término  fst (unit as Unit, $\lambda$x : (B, B). snd x) le
corresponde el siguiente árbol de derivación de tipos:



$$
\vcenter {
\infer[T-Fst] {fst\ (unit\ as\ Unit,\ \lambda x  : (B, B).\ snd\ x) : Unit } {
  \infer[T-Pair] {\vdash (unit\ as\ Unit,\ \lambda x  : (B, B).\ snd\ x):(Unit,(B,B)\to B)} {
    \infer[T-Ascribe] {\vdash unit\ as\ Unit : Unit \ } {
       \infer[T-Unit] {\vdash unit : Unit} {}
        }
    &
    \infer[T-Abs] {\vdash \lambda x  : (B, B).\ snd\ x):(B,B)\to B} {
      \infer[T-Snd] {x:(B,B) \vdash snd\ x : B} {
        \infer[T-Var] {x:(B,B) \vdash x:(B,B)} {x:(B,B) \in x:(B,B)}
        }
      }
    }
  } 
}
$$

El mismo término evaluado en el intérprete, tipa correctamente como se puede ver a continuación:

\begin{minted}{text}
ST> :type fst (unit as Unit, \x : (B, B). snd x)
Unit
\end{minted}

\newpage
\section*{\centering Codigo fuente}
\subsection*{Simplytyped.hs}
\inputminted[fontsize=\footnotesize]{haskell}{Simplytyped.hs}
\subsection*{PrettyPrinter.hs}
\inputminted[fontsize=\footnotesize]{haskell}{PrettyPrinter.hs}
\subsection*{Parse.y}
\inputminted[fontsize=\footnotesize]{haskell}{Nparse.y}
\subsection*{Common.hs}
\inputminted[fontsize=\footnotesize]{haskell}{Common.hs}
\subsection*{Ack.lam}
\inputminted[fontsize=\footnotesize]{haskell}{Ack.lam}
%\lstinputlisting{Untyped.hs}
%\lstinputlisting{Sqrt.lam}
\end{document}
